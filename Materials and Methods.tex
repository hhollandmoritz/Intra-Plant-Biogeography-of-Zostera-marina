
\subsubsection{\textbf{Field Collection and Storage}}

    All six plants were collected from \textit{Zostera marina} meadows in Bodega Bay, CA. We collected the first plant in [time/date] of 2014. The other five plants were collected in early February of 2015 as a follow-up study of the first. We stored all plants at -80 degrees prior to DNA extraction. 

\subsubsection{\textbf{Sample Preparation}}
Plant 1:
Before extraction, each plant was cut into roughly 50 pieces. The first plant was cut into forty-six pieces (summarized in the figure below). Previous data (unpublished) had shown obvious differences between tissue types, so our study focused on fine-scale community variation within each tissue type rather than among them. Briefly, leaves were cut into twenty roughly equally-sized pieces (about five centimeters each). The relative position of each piece was recorded. Similarly, the rhizome was sectioned into six segments. Each rhizome segment was between 1.5 and 2 centimeters in length and contained no more than a single root bundle.

Because we were unsure of the appropriate amount of tissue for extraction, half of the roots were sectioned into four smaller pieces while the other half were left whole. Of the whole roots, three were taken from a root bundle near the shoot, while the other three were taken from a root bundle farther out on the rhizome and pooled together into one sample. The sectioned roots were divided into four pieces we refer to as tip, mid1, mid2, and base. Similar to the whole-root samples three roots were taken from a root bundle near the plant shoot while the other three were taken from farther down on the rhizome and pooled according to plant part. 

After sectioning, pieces from plant 1 were placed in Zymo Expedition Stabilization Buffer (citation) for storage until extraction.

Plant 2-6:
For plant 2-6 a more targeted approach was taken. For each plant, a single leaf shoot was chosen and divided into roughly equally-sized pieces. Rhizomes were also divided into roughly equally sized pieces (again each piece contained no more than one root bundle). In the case of both the leaf and rhizome distance from the base of the shoot was recorded. Knowing that segmented roots would yield sufficient DNA for extraction, six roots were selected from each plant for segmentation. Three of these roots came from a root bundle closest to the leaf and three came from a root bundle farther out on the rhizome. As in plant 1, each root was divided into four equally-sized segments which we refer to as tip, mid1, mid2 and base. 

After sectioning, samples from plants 2-6 were placed directly into MoBio Powersoil bead-beating tubes and frozen at -20 degrees prior to extraction.
    